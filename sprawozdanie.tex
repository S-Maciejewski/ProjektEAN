\documentclass[a4paper, 10pt]{article}

%% Language and font encodings
\usepackage[polish]{babel}
\usepackage[utf8x]{inputenc}
\usepackage[T1]{fontenc}

\usepackage{listings}
\usepackage{xcolor}
\lstset { %
    language=C++,
    backgroundcolor=\color{black!5}, % set backgroundcolor
    basicstyle=\footnotesize,% basic font setting
}


%% Sets page size and margins
\usepackage[a4paper,top=3cm,bottom=2cm,left=3cm,right=3cm,marginparwidth=1.75cm]{geometry}

%% Useful packages
\usepackage{amsmath}
\usepackage{graphicx}
\usepackage[colorinlistoftodos]{todonotes}
\usepackage[colorlinks=true, allcolors=blue]{hyperref}
\usepackage{enumitem}

\setlength\parindent{0pt}

\def\changemargin#1#2{\list{}{\rightmargin#2\leftmargin#1}\item[]}
\let\endchangemargin=\endlist 

\title{Obliczanie wartości i współczynników funkcji sklejanej stopnia trzeciego z zadanymi
wartościami jej pochodnej na końcach przedziału}
\author{Sebastian Maciejewski \\Nr indeksu 132275}

\begin{document}
\maketitle

\section{Zastosowanie i opis funkcji}

Funkcja \textit{clampedsplinevalue} oblicza w zadanym punkcie wartość funkcji sklejanej stopnia trzeciego, która interpoluje funkcję $f$ p znanych wartościach $f(x_i)$ dla $i = 0, 1, ..., n$
oraz wartościach $f'(x_0)$ i $f'(x_n)$. \\

\noindent Procedura \textit{clampedsplinecoeffns} oblicza współczynniki przy kolejnych potęgach $x$ funkcji sklejanej trzeciego stopnia, tj. 
$$ S(x) = a_{0,i} + a_{1,i}x + a_{2,i}x^2 + a_{3,i}x^3, \qquad  i=0,1,...,n-1, $$
interpolującej funkcję $f$, dla której znane są wartości $f(x_i)$ w węzłach $x_i$ oraz $f'(x_0)$ i $f'(x_n)$.

\section{Opis metody}

\subsection*{Wyznaczanie wartości}

Funkcję sklejaną stopnia trzeciego można w każdym z przedziałów $[x_i,x_{i-1})$ przedstawić w postaci
\begin{equation}
S(x) = a_i + b_i t+c_i t^2 + d_i t^3
\end{equation}

gdzie $t=x-x_i$ dla $x \in [x_i,x_{i-1}), i=0,1,...,n-1$. W celu wyznaczenia funkcji sklejanej
postaci (1), która spełnia warunki\\

$S(x_i) = f(x_i)$ dla $i = 0,1,...,n-1$ oraz $S'(x_0) = f'(x_0)$ i $S'(x_n) = f'(x_n)$,\\ 

tj. określenia współczynników $a_i , b_i , c_i$ oraz $d_i$ dla $i = 0,1,...,n-1$ należy najpierw rozwiązać układ równań liniowych

\begin{equation}
\begin{split}
&2M_0 + \lambda_0 M_1 = \delta_0\\
&\mu_i M_{i-1} + 2M_i + \lambda_i M_{i+1} = \delta_i, \quad i= 1,2,...,n-1,\\
&\mu_n M_{i-1} + 2M_n = \delta_n,
\end{split}
\end{equation}

w którym współczynniki $\lambda_i (i=0,1,...,n-1)$, $\mu_i (i=1,2,...,n)$ i $\delta_i (i=0,1,...,n)$ są określone następującymi wzorami:
\begin{align*}
\begin{split}
&\lambda_0 = 1,\\
&\delta_0 = \frac{6}{h_1}\left(\frac{f(x_1) - f(x_0)}{h_1} - f'(x_0)\right),\\
&\lambda_i = \frac{h_{i+1}}{h_{i+1}+h_i},\\ 
&\mu_i = 1-\lambda_i ,\\
&\delta_i = \frac{6}{h_{i+1}+h_i}\left(\frac{f(x_{i+1})-f(x_i)}{h_{i+1}} - \frac{f(x_i)-f(x_{i-1})}{h_i}\right), \quad i=1,2,...,n-1,\\
&\mu_n=1,\\
&\delta_n = \frac{6}{h_n}\left(f'(x_n) - \frac{f(x_n)-f(x_{n-1})}{h_n}\right),
\end{split}
\end{align*}

przy czym $h_{i+1} = x_{x+1} - x_i$. Po rozwiązaniu układu równań (2) współczynniki funkcji sklejanej wyznacza się z następujących zależności:

\begin{align*}
\begin{split}
&a_i = f(x_i),\\
&b_i = \frac{f(x_{i+1})-f(x_i)}{h_{i+1}} - \frac{2M_i + M_{i+1}}{6}h_{i+1},\\
&c_i = \frac{M_i}{2},\\
&d_i = \frac{M_{i+1}-M_i}{6h_{i+1}}.
\end{split}
\end{align*}

W funkcji \textit{clampedsplinevalue} układ równań liniowych (2) rozwiązuje się metodą Crouta,
a następnie określa się przedział $[x_i,x_{i+1})$ zawierający dany punkt $x$ i dla tego przedziału wyznacza się współczynniki $a_i, b_i, c_i$ oraz $d_i$. Następnie z wzoru (1) oblicza się wartość funkcji sklejanej.

\subsection*{Wyznaczanie współczynników}
Współczynniki obliczane przez procedurę \textit{clampedsplinecoeffns}, tj. $a_{0,i},a_{1,i},a_{2,i}$ oraz $a_{3,i}$, są związane ze współczynnikami $a_i, b_i, c_i$ oraz $d_i$ funkcji sklejanej $S$ zapisanej w postaci (1) następującymi zależnościami: 

\begin{align*}
\begin{split}
&a_{0,i} = a_i - b_ix_i + c_i x_i^2 - d_ix_i^3, \\
&a_{1,i} = b_i - 2c_ix_i + 3d_ix_i^2, \\
&a_{2,i} = c_i - 3d_ix_i, \\
&a_{3,i} = d_i. \\
\end{split}
\end{align*}

W celu otrzymania współczynników funkcji $S$ zapisanej w postaci (1) należy skorzystać z następujących wzorów:

\begin{align*}
\begin{split}
&a_i = a_{0,i} + a_{1,i}x_i + a_{2,i}x_i^2 + a_{3,i}x_i^3, \\
&b_i = a_{1,i} + 2a_{2,i}x_i + 3a_{3,i}x_i^2, \\
&c_i = a_{2,i} + 3a_{3,i}x_i, \\
&d_i = a_{3,i}. \\
\end{split}
\end{align*}
 
\section{Wywołanie procedury}

\subsection*{Wyznaczanie wartości}

\textit{intervalclampedsplinevalue (n, x, f, f1x0, f1xn, xx, st)}

\subsection*{Wyznaczanie współczynników}

\textit{intervalclampedsplinecoeffns (n, x, f, f1x0, f1xn, a, st)}

\section{Dane}

\begin{description}
\item[$n$ \enskip --] liczba węzłów interpolacji minus 1 (węzły są ponumerowane od 0 do n),
\item[$x$ \enskip --] tablica zawierająca wartości węzłów,
\item[$f$ \enskip --] tablica zawierająca wartości interpolowanej funkcji w węzłach,
\item[$f1x0$ \enskip --] wartość $f'(x_0)$,
\item[$f1xn$ \enskip --] wartość $f'(x_n)$,
\item[$xx$ \enskip --] punkt, w którym należy obliczyć wartość naturalnej funkcji sklejanej stopnia trzeciego.
\end{description}

\section{Wyniki}

\subsection*{Wyznaczanie wartości}

\textit{clampedsplinevalue (n,x,f,f1x0,f1xn,xx,st)} - wartość funkcji sklejanej stopnia trzeciego w punkcie $xx$.

\subsection*{Wyznaczanie współczynników}

\begin{description}
\item[$a$ \enskip --] tablica współczynników funkcji sklejanej (element $a[k,i]$ zawiera wartość współczynnika przy $x^k (k=0,1,2,3)$ dla przedziału $[x_i, x_{i+1}),\enskip i=0,1,...,n-1$).
\end{description}

\section{Inne parametry}

\begin{description}
\item[$st$ \enskip --] zmienna, której w wyniku wykonania funkcji zostanie przypisana jedna z następujących wartości:

\begin{changemargin}{0,5cm}{1cm} 
1, jeżeli $n < 1$, \\
2, gdy istnieją równe wartości $x[i]$ i $x[j]$ dla $i \neq j (i,j = 0,1,...,n)$,\\
3, jeśli $xx < x[0]$ lub $xx>x[n]$, \\
0, w przeciwnym przypadku.
\end{changemargin} 

\textbf{Uwaga:} Jeżeli $st \neq 0$, to wartość funkcji \textit{clampedsplinevalue} i elementy tablicy $a$ w funkcji \textit{clampedsplinecoeffns} nie są obliczane.
\end{description}

\section{Typy parametrów}

\subsection*{Wyznaczanie wartości}

\textbf{\textit{Integer}}: $n,st$\\
\textit{interval}: $f1x0,f1xn,xx$\\
\textit{Ivector}: $x,f$

\subsection*{Wyznaczanie współczynników}

\textbf{\textit{Integer}}: $n,st$\\
\textit{interval}: $f1x0,f1xn$\\
\textit{Ivector}: $x,f$\\
\textit{Imatrix}: $a$

\section{Identyfikatory nielokalne}

\begin{description}[before={\renewcommand\makelabel[1]{\normalfont ##1}}]
\item[\textit{Ivector} --] nazwa typu tablicowego $[q_0...q_n]$ o elementach typu \textit{interval}
\item[\textit{Imatrix} --] nazwa typu tablicowego $[0..3, q_0...q_{n-1}]$ o elementach typu \textit{interval}
\end{description}


\section{Funkcje i procedury}
\begin{lstlisting}
procedure intervalclampedsplinecoeffns (n         : Integer;
                                x,f       : Ivector;
                                f1x0,f1xn : interval;
                                var a     : Imatrix;
                                var st    : Integer);

var i,k        : Integer;
    u,v,y,z,xi : interval;
    d          : Ivector;
    b          : Ivector1;
    c          : Ivector2;
begin
SetLength(b,n);
SetLength(c,n);
SetLength(d,n);

  if n<1
    then st:=1
    else begin
           st:=0;
           i:=-1;
           repeat
             i:=i+1;
             for k:=i+1 to n do
               if compare_equal(x[i],x[k])
                 then st:=2
           until (i=n-1) or (st=2)
         end;
  if st=0
    then begin
           b[0]:=1;
           u:=x[1]-x[0];
           d[0]:=6*((f[1]-f[0])/u-f1x0)/u;
           c[n]:=1;
           u:=x[n]-x[n-1];
           d[n]:=6*(f1xn-(f[n]-f[n-1])/u)/u;
           for i:=1 to n-1 do
             begin
               z:=x[i];
               y:=x[i+1]-z;
               z:=z-x[i-1];
               u:=f[i];
               b[i]:=y/(y+z);
               c[i]:=1-b[i];
               d[i]:=6*((f[i+1]-u)/y-(u-f[i-1])/z)/(y+z)
             end;
           u:=2;
           i:=-1;
           y:=d[0]/u;
           d[0]:=y;
           repeat
             i:=i+1;
             z:=b[i]/u;
             b[i]:=z;
             u:=2-z*c[i+1];
             y:=(d[i+1]-y*c[i+1])/u;
             d[i+1]:=y
           until i=n-1;
           u:=d[n];
           for i:=n-1 downto 0 do
             begin
               u:=d[i]-u*b[i];
               d[i]:=u
             end;
           for i:=0 to n-1 do
             begin
               u:=f[i];
               xi:=x[i];
               z:=x[i+1]-xi;
               y:=d[i];
               v:=(f[i+1]-u)/z-(2*y+d[i+1])*z/6;
               z:=(d[i+1]-y)/(6*z);
               y:=y/2;
               a[0,i]:=((-z*xi+y)*xi-v)*xi+u;
               u:=3*z*xi;
               a[1,i]:=(u-2*y)*xi+v;
               a[2,i]:=y-u;
               a[3,i]:=z
             end
         end
end;


function intervalclampedsplinevalue (n            : Integer;
                             x,f          : Ivector;
                             f1x0,f1xn,xx : interval;
                             var st       : Integer) : interval;

var i,k   : Integer;
    u,y,z : interval;
    found : Boolean;
    a     : array [0..3] of interval;
    d     : Ivector;
    b     : Ivector1;
    c     : Ivector2;
begin
SetLength(b,n);
SetLength(c,n);
SetLength(d,n);
  if n<1
    then st:=1
    else if (compare_less(xx,x[0])) or compare_less(x[n],xx)
           then st:=3
           else begin
                  st:=0;
                  i:=-1;
                  repeat
                    i:=i+1;
                    for k:=i+1 to n do
                      if compare_equal(x[i],x[k])
                        then st:=2
                  until (i=n-1) or (st=2)
                end;
  if st=0
    then begin
           b[0]:=1;
           u:=x[1]-x[0];
           d[0]:=6*((f[1]-f[0])/u-f1x0)/u;
           c[n]:=1;
           u:=x[n]-x[n-1];
           d[n]:=6*(f1xn-(f[n]-f[n-1])/u)/u;
           for i:=1 to n-1 do
             begin
               z:=x[i];
               y:=x[i+1]-z;
               z:=z-x[i-1];
               u:=f[i];


               b[i]:=y/(y+z);
               c[i]:=1-b[i];
               d[i]:=6*((f[i+1]-u)/y-(u-f[i-1])/z)/(y+z)
             end;
           u:=2;
           i:=-1;
           y:=d[0]/u;
           d[0]:=y;
           repeat
             i:=i+1;
             z:=b[i]/u;
             b[i]:=z;
             u:=2-z*c[i+1];
             y:=(d[i+1]-y*c[i+1])/u;
             d[i+1]:=y
           until i=n-1;
           u:=d[n];
           for i:=n-1 downto 0 do
             begin
               u:=d[i]-u*b[i];
               d[i]:=u
             end;
           found:=False;
           i:=-1;
           repeat
             i:=i+1;
             if (compare_less_or_equal(x[i],xx) and 
             (compare_less_or_equal(xx,x[i+1])))
               then found:=True
           until found;
           y:=x[i+1]-x[i];
           z:=d[i+1];
           u:=d[i];
           a[0]:=f[i];
           a[1]:=(f[i+1]-f[i])/y-(2*u+z)*y/6;
           a[2]:=u/2;
           a[3]:=(z-u)/(6*y);
           y:=a[3];
           z:=xx-x[i];
           for i:=2 downto 0 do
             y:=y*z+a[i];
           intervalclampedsplinevalue:=y
         end
end;
\end{lstlisting}

\newpage

\section{Przykłady}

\subsection*{Błędne przykłady}
\begin{lstlisting}
Dane:
n=1, x[0]=2, x[1]=3, f[0]=0, f[1]=1, f1x0=1, f1xn=1, xx=0,5

Wyniki:
st = 3


Dane:
n=1, x[0]=(1;2), x[1]=(1;2), f[0]=(3;4), f[1]=(3;4), f1x0=(5;6), f1xn=(5;6)

Wyniki:
st = 2
\end{lstlisting}

\subsection*{Zwykła arytmetyka}

\begin{lstlisting}
Obliczanie wartosci:
Dane:
n=1, x[0]=0, x[1]=1, f[0]=0, f[1]=1, f1x0=1, f1xn=1, xx=0,5

Wyniki:
Wynik=5.00000000000000E-0001,
st=0


Obliczanie wspolczynnikow:
Dane:
n=1, x[0]=0, x[1]=1, f[0]=0, f[1]=1,
f1x0=1, f1xn=1

Wyniki: 
a[0,0]=0.0000000000000000E+0000	a[1,0]=1.000000000000000E+0000
a[2,0]=0.0000000000000000E+0000	a[3,0]=0.000000000000000E+0000
st=0
\end{lstlisting}

\subsection*{Arytmetyka przedziałowa}

\begin{lstlisting}
Obliczanie wartosci:
Dane:
n=1, x[0]=(0;0), x[1]=(1;1), f[0]=(0;0), f[1]=(1;1), f1x0=(1;1), f1xn=(1;1,1),
xx=(0,5;0,6)

Wyniki:
Wynik=(4.3666666666666666E-0001, 6.6160000000000000E-0001),
Szerokosc przedzialu = 2,249E-01,
st=0


Obliczanie wspolczynnikow:
Dane:
n=1, x[0]=(0;0), x[1]=(1;1,1), f[0]=(0;0), f[1]=(1;1),
f1x0=(1;1), f1xn=(1;1)

Wyniki: 
a[0,0]=(0.0000000000000000E+0000, 0.0000000000000000E+0000)
Szerokosc przedzialu = 0.000E+00
a[1,0]=(8.0909090909090909E-0001, 1.2000000000000001E+0000)
Szerokosc przedzialu = 3.909E-01
a[2,0]=(-2.7272727272727273E-0001, 0.0000000000000000E+0000)
Szerokosc przedzialu = 2.727E-01
a[3,0]=(0.000000000000000E+0000, 1.818181818181818182E-0001)
Szerokosc przedzialu = 1.818E-01
st=0

\end{lstlisting}

\end{document}
